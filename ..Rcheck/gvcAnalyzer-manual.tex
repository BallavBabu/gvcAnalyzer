\nonstopmode{}
\documentclass[a4paper]{book}
\usepackage[times,inconsolata,hyper]{Rd}
\usepackage{makeidx}
\makeatletter\@ifl@t@r\fmtversion{2018/04/01}{}{\usepackage[utf8]{inputenc}}\makeatother
% \usepackage{graphicx} % @USE GRAPHICX@
\makeindex{}
\begin{document}
\chapter*{}
\begin{center}
{\textbf{\huge Package `gvcAnalyzer'}}
\par\bigskip{\large \today}
\end{center}
\ifthenelse{\boolean{Rd@use@hyper}}{\hypersetup{pdftitle = {gvcAnalyzer: Global Value Chain Decomposition for Value-Added Trade}}}{}
\ifthenelse{\boolean{Rd@use@hyper}}{\hypersetup{pdfauthor = {Lila Ballav Bhusal}}}{}
\begin{description}
\raggedright{}
\item[Type]\AsIs{Package}
\item[Title]\AsIs{Global Value Chain Decomposition for Value-Added Trade}
\item[Version]\AsIs{0.1.0}
\item[Author]\AsIs{Lila Ballav Bhusal [aut, cre], Alessandro Borin [ctb], Michele Mancini [ctb]}
\item[Maintainer]\AsIs{Lila Ballav Bhusal }\email{krish.bhula@gmail.com}\AsIs{}
\item[Description]\AsIs{Tools for decomposing global value chain (GVC) participation and
value-added trade using the frameworks developed by Borin and Mancini.
Implements measures and decompositions consistent with
``Measuring what matters in value-added trade''
(Economic Systems Research, 35(4), 586-613, 2023,
<}\Rhref{https://doi.org/10.1080/09535314.2022.2153221}{doi:10.1080/09535314.2022.2153221}\AsIs{>).}
\item[License]\AsIs{MIT + file LICENSE}
\item[Encoding]\AsIs{UTF-8}
\item[LazyData]\AsIs{true}
\item[Depends]\AsIs{R (>= 4.0.0)}
\item[Imports]\AsIs{Matrix, methods, stats, utils}
\item[Suggests]\AsIs{knitr, rmarkdown, testthat}
\item[VignetteBuilder]\AsIs{knitr}
\item[Roxygen]\AsIs{list(markdown = TRUE)}
\item[RoxygenNote]\AsIs{7.3.3}
\end{description}
\Rdcontents{Contents}
\HeaderA{bm\_2023\_bilateral\_pure}{BM\_2023 pure bilateral decomposition of exports from s to r}{bm.Rul.2023.Rul.bilateral.Rul.pure}
%
\begin{Description}
Implements the /sr-based decomposition:
u\_N × e\_sr = DVA\_star\_sr + DDC\_star\_sr + FVA\_star\_sr + FDC\_star\_sr
where A\textasciicircum{}/sr zeros only the bilateral intermediate block A\_sr.
\end{Description}
%
\begin{Usage}
\begin{verbatim}
bm_2023_bilateral_pure(io, s, r)
\end{verbatim}
\end{Usage}
%
\begin{Arguments}
\begin{ldescription}
\item[\code{io}] bm\_io object as returned by \code{bm\_build\_io()}

\item[\code{s}] exporter (country index or code)

\item[\code{r}] importer (country index or code)
\end{ldescription}
\end{Arguments}
%
\begin{Value}
A data frame with one row for the pair (s,r):
\begin{itemize}

\item{} \code{DVA\_star\_sr} Domestic VA of s in bilateral exports e\_sr,
when only the A\_sr block is cut.
\item{} \code{DDC\_star\_sr} Double-counting term for s in e\_sr.
\item{} \code{FVA\_star\_sr} Foreign VA in e\_sr.
\item{} \code{FDC\_star\_sr} Foreign double-counting in e\_sr.
\item{} \code{EX\_sr}       Gross exports e\_sr (scalar).

\end{itemize}

\end{Value}
\HeaderA{bm\_2023\_bilateral\_pure\_all}{BM\_2023 pure bilateral (/sr) decomposition for all pairs}{bm.Rul.2023.Rul.bilateral.Rul.pure.Rul.all}
%
\begin{Description}
Applies \code{bm\_2023\_bilateral\_pure()} to all ordered pairs (s,r), s ≠ r.
\end{Description}
%
\begin{Usage}
\begin{verbatim}
bm_2023_bilateral_pure_all(io)
\end{verbatim}
\end{Usage}
%
\begin{Arguments}
\begin{ldescription}
\item[\code{io}] bm\_io object
\end{ldescription}
\end{Arguments}
%
\begin{Value}
data.frame with one row per exporter-importer pair:
exporter, importer, DVA\_star\_sr, DDC\_star\_sr, FVA\_star\_sr, FDC\_star\_sr, EX\_sr
\end{Value}
\HeaderA{bm\_2023\_bilateral\_sink}{BM\_2023 sink-based bilateral decomposition of exports from s to r}{bm.Rul.2023.Rul.bilateral.Rul.sink}
%
\begin{Description}
Decomposes gross exports e\_sr (from exporter s to importer r) from the
importer (sink) perspective using the BM\_2023 /s formulation:
\end{Description}
%
\begin{Usage}
\begin{verbatim}
bm_2023_bilateral_sink(io, s, r)
\end{verbatim}
\end{Usage}
%
\begin{Arguments}
\begin{ldescription}
\item[\code{io}] bm\_io object as returned by \code{bm\_build\_io()}

\item[\code{s}] exporter (country index or code, e.g. 1 or "China")

\item[\code{r}] importer (country index or code, e.g. 2 or "India")
\end{ldescription}
\end{Arguments}
%
\begin{Details}
u\_N × e\_sr = DVAsink\_sr + DDCsink\_sr + FVAsink\_sr + FDCsink\_sr

where the split between \eqn{e^{(s/ \to y^*)}_{sr}}{} (ultimate shipments)
and \eqn{e^{(\to e_s^*)}_{sr}}{} (re-exports of s's exports) follows
Equations (12)–(13) in BM\_2023, constructed with the exporter-based
slash matrix A\textasciicircum{}/s (and B\textasciicircum{}/s).
\end{Details}
%
\begin{Value}
A data frame with one row for the pair (s,r):
\begin{itemize}

\item{} \code{DVAsink\_sr} Domestic VA of s in e\_sr that is finally
absorbed somewhere in the world as ultimate final demand.
\item{} \code{DDCsink\_sr} Double-counting associated with s in e\_sr
due to the circulation of s's exports along GVC chains.
\item{} \code{FVAsink\_sr} Foreign VA in e\_sr from t ≠ s.
\item{} \code{FDCsink\_sr} Foreign double-counting in e\_sr.
\item{} \code{EX\_sr}      Gross exports e\_sr (scalar).

\end{itemize}

\end{Value}
\HeaderA{bm\_2023\_bilateral\_sink\_all}{BM\_2023 sink-based bilateral decomposition for all pairs}{bm.Rul.2023.Rul.bilateral.Rul.sink.Rul.all}
%
\begin{Description}
Applies \code{bm\_2023\_bilateral\_sink()} to all ordered pairs (s,r), s ≠ r.
\end{Description}
%
\begin{Usage}
\begin{verbatim}
bm_2023_bilateral_sink_all(io)
\end{verbatim}
\end{Usage}
%
\begin{Arguments}
\begin{ldescription}
\item[\code{io}] bm\_io object
\end{ldescription}
\end{Arguments}
%
\begin{Value}
data.frame with one row per exporter-importer pair:
exporter, importer, DVAsink\_sr, DDCsink\_sr, FVAsink\_sr, FDCsink\_sr, EX\_sr
\end{Value}
\HeaderA{bm\_2023\_bilateral\_source}{BM\_2023 source-based bilateral decomposition of exports from s to r}{bm.Rul.2023.Rul.bilateral.Rul.source}
%
\begin{Description}
Decomposes gross exports e\_sr (from exporter s to importer r) as
u\_N × e\_sr = DVAsource\_sr + DDCsource\_sr + FVAsource\_sr + FDCsource\_sr
using the exporter-based slash matrix A\textasciicircum{}/s (and B\textasciicircum{}/s).
\end{Description}
%
\begin{Usage}
\begin{verbatim}
bm_2023_bilateral_source(io, s, r)
\end{verbatim}
\end{Usage}
%
\begin{Arguments}
\begin{ldescription}
\item[\code{io}] bm\_io object as returned by \code{bm\_build\_io()}

\item[\code{s}] exporter (country index or country code, e.g. 1 or "China")

\item[\code{r}] importer (country index or country code, e.g. 2 or "India")
\end{ldescription}
\end{Arguments}
%
\begin{Value}
A data frame with one row for the pair (s,r):
\begin{itemize}

\item{} \code{DVAsource\_sr} Domestic value added from s in e\_sr
that never returns to s.
\item{} \code{DDCsource\_sr} Double-counting term from s due to
multi-border domestic returns in e\_sr.
\item{} \code{FVAsource\_sr} Foreign value added in e\_sr.
\item{} \code{FDCsource\_sr} Foreign double-counting in e\_sr.
\item{} \code{EX\_sr}        Gross exports e\_sr (scalar).

\end{itemize}

\end{Value}
\HeaderA{bm\_2023\_bilateral\_source\_all}{BM\_2023 source-based bilateral decomposition for all pairs}{bm.Rul.2023.Rul.bilateral.Rul.source.Rul.all}
%
\begin{Description}
Applies \code{bm\_2023\_bilateral\_source()} to all ordered pairs (s,r), s ≠ r.
\end{Description}
%
\begin{Usage}
\begin{verbatim}
bm_2023_bilateral_source_all(io)
\end{verbatim}
\end{Usage}
%
\begin{Arguments}
\begin{ldescription}
\item[\code{io}] bm\_io object
\end{ldescription}
\end{Arguments}
%
\begin{Value}
data.frame with one row per exporter-importer pair:
exporter, importer, DVAsource\_sr, DDCsource\_sr, FVAsource\_sr, FDCsource\_sr, EX\_sr
\end{Value}
\HeaderA{bm\_2023\_exporter\_total}{BM\_2023 exporter-perspective decomposition of total exports of s}{bm.Rul.2023.Rul.exporter.Rul.total}
%
\begin{Description}
Implements Eq. (4) at the country level:
u\_N × e\_s* = DVA\_s* + DDC\_s* + FVA\_s* + FDC\_s*
\end{Description}
%
\begin{Usage}
\begin{verbatim}
bm_2023_exporter_total(io, s)
\end{verbatim}
\end{Usage}
%
\begin{Arguments}
\begin{ldescription}
\item[\code{io}] bm\_io object

\item[\code{s}] exporting country (index or code, e.g. 1 or "China")
\end{ldescription}
\end{Arguments}
%
\begin{Value}
data.frame with one row for country s
\end{Value}
\HeaderA{bm\_2023\_exporter\_total\_all}{BM\_2023 exporter totals for all countries}{bm.Rul.2023.Rul.exporter.Rul.total.Rul.all}
%
\begin{Description}
Applies \code{bm\_2023\_exporter\_total()} to every country in the IO object.
\end{Description}
%
\begin{Usage}
\begin{verbatim}
bm_2023_exporter_total_all(io)
\end{verbatim}
\end{Usage}
%
\begin{Arguments}
\begin{ldescription}
\item[\code{io}] bm\_io object
\end{ldescription}
\end{Arguments}
%
\begin{Value}
data.frame with one row per exporter:
country, DVA\_s, DDC\_s, FVA\_s, FDC\_s, EX\_s
\end{Value}
\HeaderA{bm\_2023\_trade\_components}{BM\_2023 trade-based GVC components by exporter}{bm.Rul.2023.Rul.trade.Rul.components}
%
\begin{Description}
Aggregates BM\_2023 tripartite GVC trade over all importers \eqn{r}{} for each exporter \eqn{s}{}.
\end{Description}
%
\begin{Usage}
\begin{verbatim}
bm_2023_trade_components(io)
\end{verbatim}
\end{Usage}
%
\begin{Arguments}
\begin{ldescription}
\item[\code{io}] bm\_io object
\end{ldescription}
\end{Arguments}
%
\begin{Value}
data.frame with one row per exporter and columns
\code{GVC\_sr, GVC\_PF, GVC\_TS, GVC\_PB, E\_sr}.
\end{Value}
\HeaderA{bm\_2023\_trade\_measures}{BM\_2023 trade-based GVC participation measures}{bm.Rul.2023.Rul.trade.Rul.measures}
%
\begin{Description}
From exporter-level GVC components, computes:
\begin{itemize}

\item{} \code{share\_GVC\_trade = GVC\_sr / E\_sr}
\item{} \code{share\_PF\_trade  = GVC\_PF / GVC\_sr}
\item{} \code{share\_TS\_trade  = GVC\_TS / GVC\_sr}
\item{} \code{share\_PB\_trade  = GVC\_PB / GVC\_sr}
\item{} \code{forward\_trade   = (GVC\_PF - GVC\_PB) / GVC\_sr}

\end{itemize}

\end{Description}
%
\begin{Usage}
\begin{verbatim}
bm_2023_trade_measures(io)
\end{verbatim}
\end{Usage}
%
\begin{Arguments}
\begin{ldescription}
\item[\code{io}] bm\_io object
\end{ldescription}
\end{Arguments}
%
\begin{Value}
data.frame with exporter-level trade-based participation indicators.
\end{Value}
\HeaderA{bm\_2023\_tripartite\_bilateral}{BM\_2023 tripartite GVC trade for all bilateral pairs}{bm.Rul.2023.Rul.tripartite.Rul.bilateral}
%
\begin{Description}
Computes BM\_2023 trade-based tripartite decomposition for all ordered pairs
\eqn{(s,r)}{} with \eqn{s \neq r}{}.
\end{Description}
%
\begin{Usage}
\begin{verbatim}
bm_2023_tripartite_bilateral(io)
\end{verbatim}
\end{Usage}
%
\begin{Arguments}
\begin{ldescription}
\item[\code{io}] bm\_io object
\end{ldescription}
\end{Arguments}
%
\begin{Value}
data.frame with one row per bilateral pair.
\end{Value}
\HeaderA{bm\_2023\_tripartite\_pair}{BM\_2023 tripartite GVC trade decomposition for one pair (s,r)}{bm.Rul.2023.Rul.tripartite.Rul.pair}
%
\begin{Description}
Computes BM\_2023 trade-based decomposition for exporter \eqn{s}{} and importer \eqn{r}{}:
\begin{itemize}

\item{} \code{E\_sr}: gross exports from \eqn{s}{} to \eqn{r}{}
\item{} \code{DAVAX\_sr}: domestic value added from \eqn{s}{} that crosses one border
\item{} \code{GVC\_sr}: GVC-related trade = \code{E\_sr - DAVAX\_sr}
\item{} \code{GVC\_PF}: pure-forward GVC trade
\item{} \code{GVC\_TS}: two-sided GVC trade
\item{} \code{GVC\_PB}: pure-backward GVC trade

\end{itemize}

\end{Description}
%
\begin{Usage}
\begin{verbatim}
bm_2023_tripartite_pair(io, exporter, importer)
\end{verbatim}
\end{Usage}
%
\begin{Arguments}
\begin{ldescription}
\item[\code{io}] bm\_io object

\item[\code{exporter}] exporter country (index or name)

\item[\code{importer}] importer country (index or name)
\end{ldescription}
\end{Arguments}
%
\begin{Details}
\code{exporter} and \code{importer} can be indices (1,2,...) or country names.
\end{Details}
%
\begin{Value}
data.frame with one row.
\end{Value}
\HeaderA{bm\_2025\_output\_components}{BM\_2025 output-based GVC components by exporter}{bm.Rul.2025.Rul.output.Rul.components}
%
\begin{Description}
Computes output-based PF/TS/PB GVC components for each country s.
\end{Description}
%
\begin{Usage}
\begin{verbatim}
bm_2025_output_components(io)
\end{verbatim}
\end{Usage}
%
\begin{Arguments}
\begin{ldescription}
\item[\code{io}] bm\_io object
\end{ldescription}
\end{Arguments}
%
\begin{Value}
data.frame with one row per country.
\end{Value}
\HeaderA{bm\_2025\_output\_components\_sector}{BM 2025 output components by country and sector}{bm.Rul.2025.Rul.output.Rul.components.Rul.sector}
%
\begin{Description}
This function extends \code{bm\_2025\_output\_components()} to the
country–sector level. For each country \eqn{s}{} and sector \eqn{i}{},
it allocates the country-level BM 2025 output components
(DomX, TradX, GVC\_PF\_X, GVC\_PB\_X, GVC\_TSImp, GVC\_TSDom, GVC\_TS\_X, GVC\_X)
in proportion to the sector's share in total gross output:
\end{Description}
%
\begin{Usage}
\begin{verbatim}
bm_2025_output_components_sector(io)
\end{verbatim}
\end{Usage}
%
\begin{Arguments}
\begin{ldescription}
\item[\code{io}] A \code{bm\_io} object created by \code{bm\_build\_io()}.
\end{ldescription}
\end{Arguments}
%
\begin{Details}
\deqn{
  \theta_{si} = X_{si} / \sum_{j} X_{sj},
}{}

and

\deqn{
  \mathrm{DomX}_{si}    = \theta_{si} \cdot \mathrm{DomX}_s, \quad
  \mathrm{TradX}_{si}   = \theta_{si} \cdot \mathrm{TradX}_s, \quad
  \mathrm{GVC\_PF\_X}_{si}  = \theta_{si} \cdot \mathrm{GVC\_PF\_X}_s,
}{}

etc. This ensures that for each country \eqn{s}{}:

\deqn{
  \sum_i \mathrm{DomX}_{si}    = \mathrm{DomX}_s, \quad
  \sum_i \mathrm{TradX}_{si}   = \mathrm{TradX}_s, \quad
  \sum_i \mathrm{GVC\_PF\_X}_{si} = \mathrm{GVC\\_PF\_X}_s,
}{}

and similarly for the other BM 2025 components.
\end{Details}
%
\begin{Value}
A data frame with one row per country–sector, containing:
\begin{itemize}

\item{} \code{country}: country name
\item{} \code{sector}: sector name
\item{} \code{X\_i}: sectoral gross output \eqn{X_{si}}{}
\item{} \code{DomX\_i}, \code{TradX\_i}
\item{} \code{GVC\_PF\_Xi}, \code{GVC\_PB\_Xi}
\item{} \code{GVC\_TSImp\_i}, \code{GVC\_TSDom\_i}
\item{} \code{GVC\_TS\_Xi}, \code{GVC\_Xi}

\end{itemize}

\end{Value}
\HeaderA{bm\_2025\_output\_measures}{BM\_2025 output-based GVC participation indicators}{bm.Rul.2025.Rul.output.Rul.measures}
%
\begin{Description}
For each country s:
\begin{itemize}

\item{} share\_GVC\_output = GVC\_X / X\_total
\item{} share\_PF\_output  = GVC\_PF\_X / GVC\_X
\item{} share\_TS\_output  = GVC\_TS\_X / GVC\_X
\item{} share\_PB\_output  = GVC\_PB\_X / GVC\_X
\item{} forward\_output   = (GVC\_PF\_X - GVC\_PB\_X) / GVC\_X

\end{itemize}

\end{Description}
%
\begin{Usage}
\begin{verbatim}
bm_2025_output_measures(io)
\end{verbatim}
\end{Usage}
%
\begin{Arguments}
\begin{ldescription}
\item[\code{io}] bm\_io object
\end{ldescription}
\end{Arguments}
%
\begin{Value}
data.frame with exporter-level output-based indicators.
\end{Value}
\HeaderA{bm\_2025\_output\_measures\_sector}{BM 2025 output participation measures by country and sector}{bm.Rul.2025.Rul.output.Rul.measures.Rul.sector}
%
\begin{Description}
Based on \code{bm\_2025\_output\_components\_sector()}, this function computes
sector-level BM 2025 participation measures:
\end{Description}
%
\begin{Usage}
\begin{verbatim}
bm_2025_output_measures_sector(io)
\end{verbatim}
\end{Usage}
%
\begin{Arguments}
\begin{ldescription}
\item[\code{io}] A \code{bm\_io} object created by \code{bm\_build\_io()}.
\end{ldescription}
\end{Arguments}
%
\begin{Details}
\deqn{
  \mathrm{share\_GVC\_output}_{si} = \mathrm{GVC\_X}_{si} / X_{si},
}{}

\deqn{
  \mathrm{share\_PF\_output}_{si} = \mathrm{GVC\_PF\_X}_{si} / \mathrm{GVC\_X}_{si},
}{}

\deqn{
  \mathrm{share\_TS\_output}_{si} = \mathrm{GVC\_TS\_X}_{si} / \mathrm{GVC\_X}_{si},
}{}

\deqn{
  \mathrm{share\_PB\_output}_{si} = \mathrm{GVC\_PB\_X}_{si} / \mathrm{GVC\_X}_{si},
}{}

and a sector-level forwardness index:

\deqn{
  \mathrm{forward\_output}_{si}
  = \left(\mathrm{GVC\_PF\_X}_{si} - \mathrm{GVC\_PB\_X}_{si}\right)
    / \mathrm{GVC\_X}_{si}.
}{}

Sectors with zero output or zero GVC-related output receive \code{NA}
for the corresponding ratios.
\end{Details}
%
\begin{Value}
A data frame with the columns from
\code{bm\_2025\_output\_components\_sector()} plus:
\begin{itemize}

\item{} \code{share\_GVC\_output\_i}
\item{} \code{share\_PF\_output\_i}
\item{} \code{share\_TS\_output\_i}
\item{} \code{share\_PB\_output\_i}
\item{} \code{forward\_output\_i}

\end{itemize}

\end{Value}
\HeaderA{bm\_2025\_trade\_exporter}{BM\_2025 exporter-level GVC trade totals}{bm.Rul.2025.Rul.trade.Rul.exporter}
%
\begin{Description}
Aggregates BM\_2025 tripartite trade decomposition over all destinations
for each exporter s:
\end{Description}
%
\begin{Usage}
\begin{verbatim}
bm_2025_trade_exporter(io)
\end{verbatim}
\end{Usage}
%
\begin{Arguments}
\begin{ldescription}
\item[\code{io}] bm\_io object
\end{ldescription}
\end{Arguments}
%
\begin{Details}
\eqn{E_s = sum_r E_{sr}}{},
\eqn{GVC_s = sum_r GVC_{sr}}{},
and similarly for the PF/TS/PB components.
\end{Details}
%
\begin{Value}
data.frame with one row per exporter:
exporter, E\_s, GVC\_s, GVC\_PF\_s, GVC\_TS\_s, GVC\_PB\_s.
\end{Value}
\HeaderA{bm\_2025\_trade\_measures}{BM\_2025 trade-based GVC participation indicators}{bm.Rul.2025.Rul.trade.Rul.measures}
%
\begin{Description}
Computes GVC participation and composition indicators for each exporter:
\end{Description}
%
\begin{Usage}
\begin{verbatim}
bm_2025_trade_measures(io)
\end{verbatim}
\end{Usage}
%
\begin{Arguments}
\begin{ldescription}
\item[\code{io}] bm\_io object
\end{ldescription}
\end{Arguments}
%
\begin{Details}
\begin{itemize}

\item{} share\_GVC\_trade = GVC\_s / E\_s
\item{} share\_PF\_trade  = GVC\_PF\_s / GVC\_s
\item{} share\_TS\_trade  = GVC\_TS\_s / GVC\_s
\item{} share\_PB\_trade  = GVC\_PB\_s / GVC\_s
\item{} forward\_trade   = (GVC\_PF\_s - GVC\_PB\_s) / GVC\_s

\end{itemize}

\end{Details}
%
\begin{Value}
data.frame with exporter-level totals and indicator columns.
\end{Value}
\HeaderA{bm\_2025\_tripartite\_trade}{BM\_2025 tripartite GVC trade decomposition for one pair (s,r)}{bm.Rul.2025.Rul.tripartite.Rul.trade}
%
\begin{Description}
Decompose gross exports \eqn{E_{sr}}{} (from exporter s to importer r)
into:
\end{Description}
%
\begin{Usage}
\begin{verbatim}
bm_2025_tripartite_trade(io, s, r)
\end{verbatim}
\end{Usage}
%
\begin{Arguments}
\begin{ldescription}
\item[\code{io}] bm\_io object

\item[\code{s}] exporter (country index or code, e.g. 1 or "China")

\item[\code{r}] importer (country index or code, e.g. 2 or "India")
\end{ldescription}
\end{Arguments}
%
\begin{Details}
E\_sr = DAVAX\_sr + GVC\_sr,
GVC\_sr = GVC\_PF\_sr + GVC\_TS\_sr + GVC\_PB\_sr.

This follows the BM\_2025 tripartite concept:
\begin{itemize}

\item{} DAVAX\_sr: domestic VA of s crossing one border and absorbed
in r's final demand (of s and r goods).
\item{} GVC\_PB\_sr: pure backward GVC trade (import content of
exports from s->r that stop in r's final demand).
\item{} GVC\_TS\_sr: two-sided GVC trade (imported inputs in s used to
produce exports from s->r that r uses to export further).
\item{} GVC\_PF\_sr: pure forward GVC trade, as residual:
GVC\_PF\_sr = GVC\_sr - GVC\_TS\_sr - GVC\_PB\_sr.

\end{itemize}

\end{Details}
%
\begin{Value}
A data frame with one row for the pair (s,r):
exporter, importer, E\_sr, DAVAX\_sr, GVC\_sr, GVC\_PF, GVC\_TS, GVC\_PB.
\end{Value}
\HeaderA{bm\_2025\_tripartite\_trade\_all}{BM\_2025 tripartite GVC trade decomposition for all pairs}{bm.Rul.2025.Rul.tripartite.Rul.trade.Rul.all}
%
\begin{Description}
Applies \code{bm\_2025\_tripartite\_trade()} to all ordered pairs
(s,r), s ≠ r.
\end{Description}
%
\begin{Usage}
\begin{verbatim}
bm_2025_tripartite_trade_all(io)
\end{verbatim}
\end{Usage}
%
\begin{Arguments}
\begin{ldescription}
\item[\code{io}] bm\_io object
\end{ldescription}
\end{Arguments}
%
\begin{Value}
data.frame with one row per exporter-importer pair:
exporter, importer, E\_sr, DAVAX\_sr, GVC\_sr, GVC\_PF, GVC\_TS, GVC\_PB.
\end{Value}
\HeaderA{bm\_build\_io}{Build a bm\_io object from IO table blocks}{bm.Rul.build.Rul.io}
%
\begin{Description}
Build a bm\_io object from IO table blocks
\end{Description}
%
\begin{Usage}
\begin{verbatim}
bm_build_io(Z, Y, VA, X, countries, sectors)
\end{verbatim}
\end{Usage}
%
\begin{Arguments}
\begin{ldescription}
\item[\code{Z}] Intermediate demand matrix (GN x GN), rows = suppliers, cols = users.

\item[\code{Y}] Final demand matrix (GN x G), columns = destination countries.

\item[\code{VA}] Value added vector (length GN), in same industry order as Z rows.

\item[\code{X}] Output vector (length GN), in same industry order as Z rows.

\item[\code{countries}] Character vector of country names, length G.

\item[\code{sectors}] Character vector of sector names, length N.
\end{ldescription}
\end{Arguments}
%
\begin{Value}
An object of class \code{"bm\_io"} containing
\begin{itemize}

\item{} \code{Z, Y, VA, X}
\item{} \code{countries, sectors}
\item{} \code{G, N, GN}
\item{} \code{A}: technical coefficients (GN x GN)
\item{} \code{B}: Leontief inverse (GN x GN)
\item{} \code{v}: value-added coefficients (length GN)

\end{itemize}

\end{Value}
\HeaderA{bm\_country\_id}{Resolve a country (name or index) to numeric index 1..G}{bm.Rul.country.Rul.id}
%
\begin{Description}
Resolve a country (name or index) to numeric index 1..G
\end{Description}
%
\begin{Usage}
\begin{verbatim}
bm_country_id(io, country)
\end{verbatim}
\end{Usage}
\HeaderA{bm\_davax\_sr}{DAVAX\_sr: Domestic value added absorbed in final demand in r}{bm.Rul.davax.Rul.sr}
%
\begin{Description}
Implements the DAVAX component for exports from s to r in BM\_2025
tripartite GVC trade decomposition.
\end{Description}
%
\begin{Usage}
\begin{verbatim}
bm_davax_sr(io, s, r)
\end{verbatim}
\end{Usage}
%
\begin{Arguments}
\begin{ldescription}
\item[\code{io}] A \code{bm\_io} object created by \code{bm\_build\_io()}.

\item[\code{s}] Exporting country (index or name).

\item[\code{r}] Importing country (index or name).
\end{ldescription}
\end{Arguments}
%
\begin{Details}
It corresponds (in your notation) to:
\eqn{DAVAX_{sr} = v_s L_{ss} ( y_{sr} + A_{sr} L_{rr} y_{rr} )}{}.
\end{Details}
%
\begin{Value}
Numeric scalar, DAVAX\_sr.
\end{Value}
\HeaderA{bm\_get\_e\_sr}{Exports from s to r by sector (intermediate + final)}{bm.Rul.get.Rul.e.Rul.sr}
%
\begin{Description}
Computes a length-N vector of exports from exporter \eqn{s}{} to importer \eqn{r}{},
summing intermediate and final exports:
\deqn{ e_{sr} = Z_{s,r} \mathbf{u}_N + Y_{s,r} }{}
\end{Description}
%
\begin{Usage}
\begin{verbatim}
bm_get_e_sr(io, exporter, importer)
\end{verbatim}
\end{Usage}
%
\begin{Arguments}
\begin{ldescription}
\item[\code{io}] bm\_io object

\item[\code{exporter}] exporter country (index or name)

\item[\code{importer}] importer country (index or name)
\end{ldescription}
\end{Arguments}
%
\begin{Details}
\code{exporter} and \code{importer} can be numeric indices or country names.
\end{Details}
%
\begin{Value}
Numeric vector of length N with sectoral exports s -> r.
\end{Value}
\HeaderA{bm\_get\_e\_star}{Exports from s to all foreign partners by sector (e\_s*)}{bm.Rul.get.Rul.e.Rul.star}
%
\begin{Description}
Computes a length-N vector of total exports from exporter \eqn{s}{} to all
foreign partners (intermediate + final), excluding domestic uses:
\deqn{ e_s^* = \sum_{r \neq s} e_{sr} }{}
\end{Description}
%
\begin{Usage}
\begin{verbatim}
bm_get_e_star(io, exporter)
\end{verbatim}
\end{Usage}
%
\begin{Arguments}
\begin{ldescription}
\item[\code{io}] bm\_io object

\item[\code{exporter}] exporter country (index or name)
\end{ldescription}
\end{Arguments}
%
\begin{Details}
\code{exporter} can be numeric index or country name.
\end{Details}
%
\begin{Value}
Numeric vector of length N with sectoral exports of s to all foreign partners.
\end{Value}
\HeaderA{bm\_gvc\_measures\_all}{Combined trade- and output-based GVC measures}{bm.Rul.gvc.Rul.measures.Rul.all}
%
\begin{Description}
Returns a list with:
\begin{itemize}

\item{} trade  : trade-based decomposition + indicators
\item{} output : output-based decomposition + indicators

\end{itemize}

\end{Description}
%
\begin{Usage}
\begin{verbatim}
bm_gvc_measures_all(io)
\end{verbatim}
\end{Usage}
%
\begin{Arguments}
\begin{ldescription}
\item[\code{io}] A \code{bm\_io} object.
\end{ldescription}
\end{Arguments}
%
\begin{Value}
A named list with components \code{trade} and \code{output}.
\end{Value}
\HeaderA{bm\_gvc\_output\_all}{Output-based GVC decomposition for all countries}{bm.Rul.gvc.Rul.output.Rul.all}
%
\begin{Description}
Output-based GVC decomposition for all countries
\end{Description}
%
\begin{Usage}
\begin{verbatim}
bm_gvc_output_all(io)
\end{verbatim}
\end{Usage}
%
\begin{Arguments}
\begin{ldescription}
\item[\code{io}] A \code{bm\_io} object.
\end{ldescription}
\end{Arguments}
%
\begin{Value}
A \code{data.frame} with one row per country.
\end{Value}
\HeaderA{bm\_gvc\_output\_country}{Output-based GVC decomposition for one country}{bm.Rul.gvc.Rul.output.Rul.country}
%
\begin{Description}
For a given country s, this function returns a row with:
\begin{itemize}

\item{} GVC\_PF\_X  : pure-forward GVC-related output
\item{} GVC\_PB\_X  : pure-backward GVC-related output
\item{} GVC\_TSImp : imported-input two-sided GVC-related output
\item{} GVC\_TSDom : domestic-input two-sided GVC-related output
\item{} GVC\_TS\_X  : total two-sided (sum of imported and domestic)
\item{} GVC\_X     : total GVC-related output (PF + PB + TS)
\item{} DomX      : proxy for purely domestic output chains
\item{} TradX     : residual "traditional" (one-border) trade-related output
\item{} X\_total   : total output of s

\end{itemize}

\end{Description}
%
\begin{Usage}
\begin{verbatim}
bm_gvc_output_country(io, s, Xexp_list = NULL)
\end{verbatim}
\end{Usage}
%
\begin{Arguments}
\begin{ldescription}
\item[\code{io}] A \code{bm\_io} object.

\item[\code{s}] Country (index or name).

\item[\code{Xexp\_list}] Optional precomputed export-related outputs list.
\end{ldescription}
\end{Arguments}
%
\begin{Value}
A one-row \code{data.frame} with these components.
\end{Value}
\HeaderA{bm\_gvc\_output\_measures}{Output-based GVC participation indicators}{bm.Rul.gvc.Rul.output.Rul.measures}
%
\begin{Description}
Builds on \code{bm\_gvc\_output\_all()} and computes, for each country:
\begin{itemize}

\item{} share\_GVC\_output = GVC\_X / X\_total
\item{} share\_PF\_output  = GVC\_PF\_X / GVC\_X
\item{} share\_TS\_output  = GVC\_TS\_X / GVC\_X
\item{} share\_PB\_output  = GVC\_PB\_X / GVC\_X
\item{} forward\_output   = (GVC\_PF\_X - GVC\_PB\_X) / GVC\_X

\end{itemize}

\end{Description}
%
\begin{Usage}
\begin{verbatim}
bm_gvc_output_measures(io)
\end{verbatim}
\end{Usage}
%
\begin{Arguments}
\begin{ldescription}
\item[\code{io}] A \code{bm\_io} object.
\end{ldescription}
\end{Arguments}
%
\begin{Details}
These mirror the trade-based indicators on the output side.
\end{Details}
%
\begin{Value}
A \code{data.frame} with one row per country.
\end{Value}
\HeaderA{bm\_gvc\_pb\_output}{Pure-backward GVC-related output (country level)}{bm.Rul.gvc.Rul.pb.Rul.output}
%
\begin{Description}
For a given country s, this function computes the pure-backward
GVC-related output, interpreted as foreign value added embodied
in s's output that is ultimately absorbed abroad in a one-border
GVC pattern.
\end{Description}
%
\begin{Usage}
\begin{verbatim}
bm_gvc_pb_output(io, s)
\end{verbatim}
\end{Usage}
%
\begin{Arguments}
\begin{ldescription}
\item[\code{io}] A \code{bm\_io} object.

\item[\code{s}] Country (index or name).
\end{ldescription}
\end{Arguments}
%
\begin{Details}
This follows the BM\_2025 style expression using:
\begin{itemize}

\item{} Y\_s\_total = Σ\_z Y\_sz
\item{} Y\_ss      = Y\_ss
\item{} A\_jk, B\_ks, L\_jj, L\_ss blocks.

\end{itemize}

\end{Details}
%
\begin{Value}
Numeric scalar: pure-backward GVC-related output of s.
\end{Value}
\HeaderA{bm\_gvc\_pf\_output}{Pure-forward GVC-related output (country level)}{bm.Rul.gvc.Rul.pf.Rul.output}
%
\begin{Description}
For a given country s, this function computes the pure-forward
GVC-related output, i.e. domestic value added of s embodied in r's
export-related output to third countries (forward chains).
\end{Description}
%
\begin{Usage}
\begin{verbatim}
bm_gvc_pf_output(io, s, Xexp_list = NULL)
\end{verbatim}
\end{Usage}
%
\begin{Arguments}
\begin{ldescription}
\item[\code{io}] A \code{bm\_io} object.

\item[\code{s}] Country (index or name).

\item[\code{Xexp\_list}] Optional precomputed list of export-related output
(as from \code{bm\_build\_Xexp\_list(io)}). If \code{NULL}, it is
built internally.
\end{ldescription}
\end{Arguments}
%
\begin{Value}
Numeric scalar: pure-forward GVC-related output of s.
\end{Value}
\HeaderA{bm\_gvc\_trade\_by\_exporter}{Trade-based GVC decomposition by exporter}{bm.Rul.gvc.Rul.trade.Rul.by.Rul.exporter}
%
\begin{Description}
Aggregates the bilateral tripartite GVC trade decomposition
over all importing partners for each exporting country s.
\end{Description}
%
\begin{Usage}
\begin{verbatim}
bm_gvc_trade_by_exporter(io)
\end{verbatim}
\end{Usage}
%
\begin{Arguments}
\begin{ldescription}
\item[\code{io}] A \code{bm\_io} object.
\end{ldescription}
\end{Arguments}
%
\begin{Details}
For each exporter, it returns:
\begin{itemize}

\item{} E\_sr     : total gross exports (sum over r)
\item{} GVC\_sr   : total GVC-related trade
\item{} GVC\_PF   : pure-forward GVC-related trade
\item{} GVC\_TS   : two-sided GVC-related trade
\item{} GVC\_PB   : pure-backward GVC-related trade

\end{itemize}

\end{Details}
%
\begin{Value}
A \code{data.frame} with one row per exporter.
\end{Value}
\HeaderA{bm\_gvc\_trade\_measures}{Trade-based GVC participation indicators}{bm.Rul.gvc.Rul.trade.Rul.measures}
%
\begin{Description}
Builds on \code{bm\_gvc\_trade\_by\_exporter()} and computes,
for each exporter:
\begin{itemize}

\item{} share\_GVC\_trade = GVC\_sr / E\_sr
\item{} share\_PF\_trade  = GVC\_PF / GVC\_sr
\item{} share\_TS\_trade  = GVC\_TS / GVC\_sr
\item{} share\_PB\_trade  = GVC\_PB / GVC\_sr
\item{} forward\_trade   = (GVC\_PF - GVC\_PB) / GVC\_sr

\end{itemize}

\end{Description}
%
\begin{Usage}
\begin{verbatim}
bm_gvc_trade_measures(io)
\end{verbatim}
\end{Usage}
%
\begin{Arguments}
\begin{ldescription}
\item[\code{io}] A \code{bm\_io} object.
\end{ldescription}
\end{Arguments}
%
\begin{Details}
These correspond to the paper's trade-based participation
and forwardness indicators.
\end{Details}
%
\begin{Value}
A \code{data.frame} with one row per country.
\end{Value}
\HeaderA{bm\_gvc\_ts\_domestic\_output}{Two-sided GVC-related output from domestic inputs}{bm.Rul.gvc.Rul.ts.Rul.domestic.Rul.output}
%
\begin{Description}
Domestic two-sided GVC-related output of s is the domestic part of
two-sided GVC chains: domestic value added of s embodied in r's GVC
production that combines domestic and foreign inputs.
\end{Description}
%
\begin{Usage}
\begin{verbatim}
bm_gvc_ts_domestic_output(io, s, Xexp_list = NULL)
\end{verbatim}
\end{Usage}
%
\begin{Arguments}
\begin{ldescription}
\item[\code{io}] A \code{bm\_io} object.

\item[\code{s}] Country (index or name).

\item[\code{Xexp\_list}] Optional precomputed export-related outputs list.
\end{ldescription}
\end{Arguments}
%
\begin{Value}
Numeric scalar: domestic two-sided GVC-related output of s.
\end{Value}
\HeaderA{bm\_gvc\_ts\_import\_output}{Two-sided GVC-related output from imported inputs}{bm.Rul.gvc.Rul.ts.Rul.import.Rul.output}
%
\begin{Description}
Imported two-sided GVC-related output of s is defined as all imported
inputs used in s's output, net of those absorbed in purely domestic
chains, net of the pure-backward part.
\end{Description}
%
\begin{Usage}
\begin{verbatim}
bm_gvc_ts_import_output(io, s, gvc_pb_s)
\end{verbatim}
\end{Usage}
%
\begin{Arguments}
\begin{ldescription}
\item[\code{io}] A \code{bm\_io} object.

\item[\code{s}] Country (index or name).

\item[\code{gvc\_pb\_s}] Pre-computed pure-backward GVC output of s
(from \code{bm\_gvc\_pb\_output}).
\end{ldescription}
\end{Arguments}
%
\begin{Value}
Numeric scalar: imported two-sided GVC-related output of s.
\end{Value}
\HeaderA{bm\_pb\_sr}{PB\_sr: Pure-backward GVC trade (import content used in r's final demand)}{bm.Rul.pb.Rul.sr}
%
\begin{Description}
Pure-backward GVC trade is the import content of production in s
that serves r's final demand of s and r goods.
\end{Description}
%
\begin{Usage}
\begin{verbatim}
bm_pb_sr(io, s, r)
\end{verbatim}
\end{Usage}
%
\begin{Arguments}
\begin{ldescription}
\item[\code{io}] A \code{bm\_io} object.

\item[\code{s}] Exporting country (index or name).

\item[\code{r}] Importing country (index or name).
\end{ldescription}
\end{Arguments}
%
\begin{Details}
In your notation:
\begin{itemize}

\item{} q\_s(sr) = L\_ss ( y\_sr + A\_sr L\_rr y\_rr )
\item{} PB\_sr = Σ\_t≠s ( u\_N A\_ts q\_s(sr) ).

\end{itemize}

\end{Details}
%
\begin{Value}
Numeric scalar, PB\_sr.
\end{Value}
\HeaderA{bm\_sector\_id}{Resolve a sector (name or index) to numeric index 1..N}{bm.Rul.sector.Rul.id}
%
\begin{Description}
Resolve a sector (name or index) to numeric index 1..N
\end{Description}
%
\begin{Usage}
\begin{verbatim}
bm_sector_id(io, sector)
\end{verbatim}
\end{Usage}
\HeaderA{bm\_toy\_data}{Toy 4-country, 3-sector IO table for bmGVC}{bm.Rul.toy.Rul.data}
\aliasA{bm\_toy\_countries}{bm\_toy\_data}{bm.Rul.toy.Rul.countries}
\aliasA{bm\_toy\_sectors}{bm\_toy\_data}{bm.Rul.toy.Rul.sectors}
\aliasA{bm\_toy\_VA}{bm\_toy\_data}{bm.Rul.toy.Rul.VA}
\aliasA{bm\_toy\_X}{bm\_toy\_data}{bm.Rul.toy.Rul.X}
\aliasA{bm\_toy\_Y}{bm\_toy\_data}{bm.Rul.toy.Rul.Y}
\aliasA{bm\_toy\_Z}{bm\_toy\_data}{bm.Rul.toy.Rul.Z}
\keyword{datasets}{bm\_toy\_data}
%
\begin{Description}
A small multi-country input–output data set used in bmGVC examples
and vignettes. It contains four countries (China, India, Japan, ROW)
and three sectors (Primary, Manufacturing, Service).
\end{Description}
%
\begin{Format}
\begin{description}

\item[bm\_toy\_Z] numeric matrix \code{12 x 12}
\item[bm\_toy\_Y] numeric matrix \code{12 x 4}
\item[bm\_toy\_VA] numeric vector of length 12
\item[bm\_toy\_X] numeric vector of length 12
\item[bm\_toy\_countries] character vector of length 4
\item[bm\_toy\_sectors] character vector of length 3

\end{description}

\end{Format}
%
\begin{Details}
The data are stored in six objects:
\begin{itemize}

\item{} \code{bm\_toy\_Z}: 12 x 12 intermediate demand matrix
\item{} \code{bm\_toy\_Y}: 12 x 4 final demand matrix
\item{} \code{bm\_toy\_VA}: length-12 value-added vector
\item{} \code{bm\_toy\_X}: length-12 gross output vector
\item{} \code{bm\_toy\_countries}: character vector of length 4
\item{} \code{bm\_toy\_sectors}: character vector of length 3

\end{itemize}


The ordering of industries is
(China P,M,S; India P,M,S; Japan P,M,S; ROW P,M,S).
\end{Details}
\HeaderA{bm\_tripartite\_trade\_all}{Tripartite GVC trade for all bilateral pairs}{bm.Rul.tripartite.Rul.trade.Rul.all}
%
\begin{Description}
Computes the BM\_2025 tripartite GVC trade decomposition for all
ordered pairs (s,r) with s ≠ r.
\end{Description}
%
\begin{Usage}
\begin{verbatim}
bm_tripartite_trade_all(io)
\end{verbatim}
\end{Usage}
%
\begin{Arguments}
\begin{ldescription}
\item[\code{io}] A \code{bm\_io} object.
\end{ldescription}
\end{Arguments}
%
\begin{Value}
A \code{data.frame} with one row per pair (s,r).
\end{Value}
\HeaderA{bm\_tripartite\_trade\_sr}{Tripartite GVC trade decomposition for a bilateral pair (s,r)}{bm.Rul.tripartite.Rul.trade.Rul.sr}
%
\begin{Description}
For exports from s to r, this function returns:
\begin{itemize}

\item{} E\_sr: gross exports (sum of e\_sr)
\item{} DAVAX\_sr: domestic value added that crosses only one border and is absorbed in r's final demand
\item{} GVC\_sr: GVC-related part of E\_sr, GVC\_sr = E\_sr - DAVAX\_sr
\item{} GVC\_PF: pure-forward GVC trade (residual)
\item{} GVC\_TS: two-sided GVC trade
\item{} GVC\_PB: pure-backward GVC trade

\end{itemize}

\end{Description}
%
\begin{Usage}
\begin{verbatim}
bm_tripartite_trade_sr(io, s, r)
\end{verbatim}
\end{Usage}
%
\begin{Arguments}
\begin{ldescription}
\item[\code{io}] A \code{bm\_io} object.

\item[\code{s}] Exporting country (index or name).

\item[\code{r}] Importing country (index or name).
\end{ldescription}
\end{Arguments}
%
\begin{Details}
In your notation:
\deqn{E_{sr} = DAVAX_{sr} + GVC_{sr}}{}
\deqn{GVC_{sr} = GVC^{PF}_{sr} + GVC^{TS}_{sr} + GVC^{PB}_{sr}}{}.
\end{Details}
%
\begin{Value}
A one-row \code{data.frame} with all components.
\end{Value}
\HeaderA{bm\_ts\_sr}{TS\_sr: Two-sided GVC trade for exports from s to r}{bm.Rul.ts.Rul.sr}
%
\begin{Description}
Two-sided GVC trade is the import content in s used to serve r's exports
to third countries (forward-plus-backward chains).
\end{Description}
%
\begin{Usage}
\begin{verbatim}
bm_ts_sr(io, s, r)
\end{verbatim}
\end{Usage}
%
\begin{Arguments}
\begin{ldescription}
\item[\code{io}] A \code{bm\_io} object.

\item[\code{s}] Exporting country (index or name).

\item[\code{r}] Importing country (index or name).
\end{ldescription}
\end{Arguments}
%
\begin{Details}
In your notation:
\begin{itemize}

\item{} sum\_e\_r\_to\_others = Σ\_j≠r e\_rj
\item{} demand in s: A\_sr L\_rr sum\_e\_r\_to\_others
\item{} q\_s(ts) = L\_ss A\_sr L\_rr sum\_e\_r\_to\_others
\item{} TS\_sr = Σ\_t≠s ( u\_N A\_ts q\_s(ts) ).

\end{itemize}

\end{Details}
%
\begin{Value}
Numeric scalar, TS\_sr.
\end{Value}
\printindex{}
\end{document}
